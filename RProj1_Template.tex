% Options for packages loaded elsewhere
\PassOptionsToPackage{unicode}{hyperref}
\PassOptionsToPackage{hyphens}{url}
%
\documentclass[
]{article}
\usepackage{lmodern}
\usepackage{amssymb,amsmath}
\usepackage{ifxetex,ifluatex}
\ifnum 0\ifxetex 1\fi\ifluatex 1\fi=0 % if pdftex
  \usepackage[T1]{fontenc}
  \usepackage[utf8]{inputenc}
  \usepackage{textcomp} % provide euro and other symbols
\else % if luatex or xetex
  \usepackage{unicode-math}
  \defaultfontfeatures{Scale=MatchLowercase}
  \defaultfontfeatures[\rmfamily]{Ligatures=TeX,Scale=1}
\fi
% Use upquote if available, for straight quotes in verbatim environments
\IfFileExists{upquote.sty}{\usepackage{upquote}}{}
\IfFileExists{microtype.sty}{% use microtype if available
  \usepackage[]{microtype}
  \UseMicrotypeSet[protrusion]{basicmath} % disable protrusion for tt fonts
}{}
\makeatletter
\@ifundefined{KOMAClassName}{% if non-KOMA class
  \IfFileExists{parskip.sty}{%
    \usepackage{parskip}
  }{% else
    \setlength{\parindent}{0pt}
    \setlength{\parskip}{6pt plus 2pt minus 1pt}}
}{% if KOMA class
  \KOMAoptions{parskip=half}}
\makeatother
\usepackage{xcolor}
\IfFileExists{xurl.sty}{\usepackage{xurl}}{} % add URL line breaks if available
\IfFileExists{bookmark.sty}{\usepackage{bookmark}}{\usepackage{hyperref}}
\hypersetup{
  pdftitle={R Project 1 - Hello R},
  pdfauthor={Aziz Surani},
  hidelinks,
  pdfcreator={LaTeX via pandoc}}
\urlstyle{same} % disable monospaced font for URLs
\usepackage[margin=1in]{geometry}
\usepackage{color}
\usepackage{fancyvrb}
\newcommand{\VerbBar}{|}
\newcommand{\VERB}{\Verb[commandchars=\\\{\}]}
\DefineVerbatimEnvironment{Highlighting}{Verbatim}{commandchars=\\\{\}}
% Add ',fontsize=\small' for more characters per line
\usepackage{framed}
\definecolor{shadecolor}{RGB}{248,248,248}
\newenvironment{Shaded}{\begin{snugshade}}{\end{snugshade}}
\newcommand{\AlertTok}[1]{\textcolor[rgb]{0.94,0.16,0.16}{#1}}
\newcommand{\AnnotationTok}[1]{\textcolor[rgb]{0.56,0.35,0.01}{\textbf{\textit{#1}}}}
\newcommand{\AttributeTok}[1]{\textcolor[rgb]{0.77,0.63,0.00}{#1}}
\newcommand{\BaseNTok}[1]{\textcolor[rgb]{0.00,0.00,0.81}{#1}}
\newcommand{\BuiltInTok}[1]{#1}
\newcommand{\CharTok}[1]{\textcolor[rgb]{0.31,0.60,0.02}{#1}}
\newcommand{\CommentTok}[1]{\textcolor[rgb]{0.56,0.35,0.01}{\textit{#1}}}
\newcommand{\CommentVarTok}[1]{\textcolor[rgb]{0.56,0.35,0.01}{\textbf{\textit{#1}}}}
\newcommand{\ConstantTok}[1]{\textcolor[rgb]{0.00,0.00,0.00}{#1}}
\newcommand{\ControlFlowTok}[1]{\textcolor[rgb]{0.13,0.29,0.53}{\textbf{#1}}}
\newcommand{\DataTypeTok}[1]{\textcolor[rgb]{0.13,0.29,0.53}{#1}}
\newcommand{\DecValTok}[1]{\textcolor[rgb]{0.00,0.00,0.81}{#1}}
\newcommand{\DocumentationTok}[1]{\textcolor[rgb]{0.56,0.35,0.01}{\textbf{\textit{#1}}}}
\newcommand{\ErrorTok}[1]{\textcolor[rgb]{0.64,0.00,0.00}{\textbf{#1}}}
\newcommand{\ExtensionTok}[1]{#1}
\newcommand{\FloatTok}[1]{\textcolor[rgb]{0.00,0.00,0.81}{#1}}
\newcommand{\FunctionTok}[1]{\textcolor[rgb]{0.00,0.00,0.00}{#1}}
\newcommand{\ImportTok}[1]{#1}
\newcommand{\InformationTok}[1]{\textcolor[rgb]{0.56,0.35,0.01}{\textbf{\textit{#1}}}}
\newcommand{\KeywordTok}[1]{\textcolor[rgb]{0.13,0.29,0.53}{\textbf{#1}}}
\newcommand{\NormalTok}[1]{#1}
\newcommand{\OperatorTok}[1]{\textcolor[rgb]{0.81,0.36,0.00}{\textbf{#1}}}
\newcommand{\OtherTok}[1]{\textcolor[rgb]{0.56,0.35,0.01}{#1}}
\newcommand{\PreprocessorTok}[1]{\textcolor[rgb]{0.56,0.35,0.01}{\textit{#1}}}
\newcommand{\RegionMarkerTok}[1]{#1}
\newcommand{\SpecialCharTok}[1]{\textcolor[rgb]{0.00,0.00,0.00}{#1}}
\newcommand{\SpecialStringTok}[1]{\textcolor[rgb]{0.31,0.60,0.02}{#1}}
\newcommand{\StringTok}[1]{\textcolor[rgb]{0.31,0.60,0.02}{#1}}
\newcommand{\VariableTok}[1]{\textcolor[rgb]{0.00,0.00,0.00}{#1}}
\newcommand{\VerbatimStringTok}[1]{\textcolor[rgb]{0.31,0.60,0.02}{#1}}
\newcommand{\WarningTok}[1]{\textcolor[rgb]{0.56,0.35,0.01}{\textbf{\textit{#1}}}}
\usepackage{graphicx,grffile}
\makeatletter
\def\maxwidth{\ifdim\Gin@nat@width>\linewidth\linewidth\else\Gin@nat@width\fi}
\def\maxheight{\ifdim\Gin@nat@height>\textheight\textheight\else\Gin@nat@height\fi}
\makeatother
% Scale images if necessary, so that they will not overflow the page
% margins by default, and it is still possible to overwrite the defaults
% using explicit options in \includegraphics[width, height, ...]{}
\setkeys{Gin}{width=\maxwidth,height=\maxheight,keepaspectratio}
% Set default figure placement to htbp
\makeatletter
\def\fps@figure{htbp}
\makeatother
\setlength{\emergencystretch}{3em} % prevent overfull lines
\providecommand{\tightlist}{%
  \setlength{\itemsep}{0pt}\setlength{\parskip}{0pt}}
\setcounter{secnumdepth}{-\maxdimen} % remove section numbering

\title{R Project 1 - Hello R}
\author{Aziz Surani}
\date{1/23/2020}

\begin{document}
\maketitle

\hypertarget{load-packages}{%
\subsubsection{Load packages}\label{load-packages}}

\begin{Shaded}
\begin{Highlighting}[]
\KeywordTok{library}\NormalTok{(tidyverse) }
\KeywordTok{library}\NormalTok{(datasauRus)}
\end{Highlighting}
\end{Shaded}

\hypertarget{exercise-1}{%
\subsubsection{Exercise 1}\label{exercise-1}}

the dataset has 3 columns. A data frame with 1846 rows and 3 variables
namely dataset, X: x-values, Y: y-values.

\hypertarget{exercise-2}{%
\subsubsection{Exercise 2}\label{exercise-2}}

First let's plot the data in the dino dataset:

\begin{Shaded}
\begin{Highlighting}[]
\NormalTok{dino_data <-}\StringTok{ }\NormalTok{datasaurus_dozen }\OperatorTok
\StringTok{  }\KeywordTok{filter}\NormalTok{(dataset }\OperatorTok{==}\StringTok{ "dino"}\NormalTok{)}
\KeywordTok{ggplot}\NormalTok{(}\DataTypeTok{data =}\NormalTok{ dino_data, }\DataTypeTok{mapping =} \KeywordTok{aes}\NormalTok{(}\DataTypeTok{x =}\NormalTok{ x, }\DataTypeTok{y =}\NormalTok{ y)) }\OperatorTok{+}
\StringTok{  }\KeywordTok{geom_point}\NormalTok{()}
\end{Highlighting}
\end{Shaded}

\includegraphics{RProj1_Template_files/figure-latex/plot-dino-1.pdf}

And next calculate the correlation between \texttt{x} and \texttt{y} in
this dataset:

\begin{Shaded}
\begin{Highlighting}[]
\NormalTok{dino_data }\OperatorTok
\StringTok{  }\KeywordTok{summarize}\NormalTok{(}\DataTypeTok{r =} \KeywordTok{cor}\NormalTok{(x, y))}
\end{Highlighting}
\end{Shaded}

\begin{verbatim}
## # A tibble: 1 x 1
##         r
##     <dbl>
## 1 -0.0645
\end{verbatim}

\hypertarget{exercise-3}{%
\subsubsection{Exercise 3}\label{exercise-3}}

First let's plot the data in the star dataset:

\begin{Shaded}
\begin{Highlighting}[]
\NormalTok{   star_data <-}\StringTok{ }\NormalTok{datasaurus_dozen }\OperatorTok
\StringTok{   }\KeywordTok{filter}\NormalTok{(dataset }\OperatorTok{==}\StringTok{ "star"}\NormalTok{)}
   \KeywordTok{ggplot}\NormalTok{(}\DataTypeTok{data =}\NormalTok{ star_data, }\DataTypeTok{mapping =} \KeywordTok{aes}\NormalTok{(}\DataTypeTok{x =}\NormalTok{ x, }\DataTypeTok{y =}\NormalTok{ y)) }\OperatorTok{+}
\StringTok{   }\KeywordTok{geom_point}\NormalTok{()}
\end{Highlighting}
\end{Shaded}

\includegraphics{RProj1_Template_files/figure-latex/plot-star-1.pdf}

And next calculate the correlation between \texttt{x} and \texttt{y} in
this dataset:

\begin{Shaded}
\begin{Highlighting}[]
\NormalTok{star_data }\OperatorTok
\StringTok{  }\KeywordTok{summarize}\NormalTok{(}\DataTypeTok{r =} \KeywordTok{cor}\NormalTok{(x, y))}
\end{Highlighting}
\end{Shaded}

\begin{verbatim}
## # A tibble: 1 x 1
##         r
##     <dbl>
## 1 -0.0630
\end{verbatim}

\hypertarget{exercise-4}{%
\subsubsection{Exercise 4}\label{exercise-4}}

\begin{Shaded}
\begin{Highlighting}[]
\NormalTok{  circle_data <-}\StringTok{ }\NormalTok{datasaurus_dozen }\OperatorTok
\StringTok{  }\KeywordTok{filter}\NormalTok{(dataset }\OperatorTok{==}\StringTok{ "circle"}\NormalTok{)}
  \KeywordTok{ggplot}\NormalTok{(}\DataTypeTok{data =}\NormalTok{ circle_data, }\DataTypeTok{mapping =} \KeywordTok{aes}\NormalTok{(}\DataTypeTok{x =}\NormalTok{ x, }\DataTypeTok{y =}\NormalTok{ y)) }\OperatorTok{+}
\StringTok{  }\KeywordTok{geom_point}\NormalTok{()}
\end{Highlighting}
\end{Shaded}

\includegraphics{RProj1_Template_files/figure-latex/plot-circle-1.pdf}

\begin{Shaded}
\begin{Highlighting}[]
\NormalTok{circle_data }\OperatorTok
\StringTok{  }\KeywordTok{summarize}\NormalTok{(}\DataTypeTok{r =} \KeywordTok{cor}\NormalTok{(x, y))}
\end{Highlighting}
\end{Shaded}

\begin{verbatim}
## # A tibble: 1 x 1
##         r
##     <dbl>
## 1 -0.0683
\end{verbatim}

\hypertarget{exercise-5}{%
\subsubsection{Exercise 5}\label{exercise-5}}

\begin{Shaded}
\begin{Highlighting}[]
  \KeywordTok{ggplot}\NormalTok{(datasaurus_dozen, }\KeywordTok{aes}\NormalTok{(}\DataTypeTok{x =}\NormalTok{ x, }\DataTypeTok{y =}\NormalTok{ y, }\DataTypeTok{color =}\NormalTok{ dataset))}\OperatorTok{+}
\StringTok{  }\KeywordTok{geom_point}\NormalTok{()}\OperatorTok{+}
\StringTok{  }\KeywordTok{facet_wrap}\NormalTok{(}\OperatorTok{~}\StringTok{ }\NormalTok{dataset, }\DataTypeTok{ncol =} \DecValTok{3}\NormalTok{) }\OperatorTok{+}\StringTok{ }\CommentTok{# facet the dataset variable, in 3 cols}
\StringTok{  }\KeywordTok{theme}\NormalTok{(}\DataTypeTok{legend.position =} \StringTok{"none"}\NormalTok{)}
\end{Highlighting}
\end{Shaded}

\includegraphics{RProj1_Template_files/figure-latex/plot-comparision-1.pdf}

\begin{Shaded}
\begin{Highlighting}[]
\NormalTok{datasaurus_dozen }\OperatorTok
\StringTok{  }\KeywordTok{group_by}\NormalTok{(dataset) }\OperatorTok
\StringTok{  }\KeywordTok{summarize}\NormalTok{(}\DataTypeTok{r =} \KeywordTok{cor}\NormalTok{(x, y)) }\OperatorTok
\StringTok{  }\KeywordTok{print}\NormalTok{(}\DecValTok{13}\NormalTok{)}
\end{Highlighting}
\end{Shaded}

\begin{verbatim}
## # A tibble: 13 x 2
##    dataset          r
##    <chr>        <dbl>
##  1 away       -0.0641
##  2 bullseye   -0.0686
##  3 circle     -0.0683
##  4 dino       -0.0645
##  5 dots       -0.0603
##  6 h_lines    -0.0617
##  7 high_lines -0.0685
##  8 slant_down -0.0690
##  9 slant_up   -0.0686
## 10 star       -0.0630
## 11 v_lines    -0.0694
## 12 wide_lines -0.0666
## 13 x_shape    -0.0656
\end{verbatim}

\end{document}
